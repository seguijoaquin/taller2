Travis Master Status \href{https://travis-ci.org/seguijoaquin/taller2-appserver.svg?branch=master}{\tt }

\href{https://travis-ci.org/seguijoaquin/taller2-appserver}{\tt Travis}

\subsection*{Descripcion}

Se trata de una aplicación linux por consola destinada a mantenerse en ejecución por períodos prolongados de tiempo. Brinda una interfaz para la comunicación de los diferentes clientes que se conecten a la misma.

La interfaz de comunicación se realiza a traves de una Restful A\+PI, que define la forma de las solicitudes y respuestas de los diferentes servicios que brindará el servidor. Estos ultimos son\+:

\subsubsection*{Servicio de autenticación}

El servidor dispondrá de un servicio para la autenticación de los clientes. Este servicio consta de una solicitud de autenticación, que viene junto con las credenciales del usuario. La respuesta a la solicitud es un token (identificador) de la sesión del usuario.

\subsubsection*{Servicio de búsqueda de candidato}

El servidor provee un servicio en donde el usuario podrá solicitar un posible candidato para match.

\subsubsection*{Servicio de conversaciones}

El servidor brindará un servicio para el delivery de conversaciones.

Para poder iniciar una conversación entre dos usuarios es necesario que exista un match entre ellos.

\subsection*{Librerias externas}

Base de datos Rocks\+DB, para instalarlo\+:


\begin{DoxyCode}
1 > ./instalar\_rocksdb.sh
\end{DoxyCode}


Framework de testing Google tests, para instalarlo\+:


\begin{DoxyCode}
1 > ./instalar\_googletest.sh
\end{DoxyCode}


(se instala como una biblioteca dinamica)

Se utiliza mongoose para la parte de web server; no es necesaria su instalacion.

Se utiliza jsoncpp como parser; no es necesaria su instalacion previa.

\subsection*{Instalacion y utilizacion}

\subsubsection*{Instalacion\+:}


\begin{DoxyCode}
1 > cd Appserver
2 > mkdir build
3 > cd build
4 > cmake ../src
5 > make
\end{DoxyCode}


\subsubsection*{Utilizacion\+:}


\begin{DoxyCode}
1 > ./Appserver
\end{DoxyCode}
 